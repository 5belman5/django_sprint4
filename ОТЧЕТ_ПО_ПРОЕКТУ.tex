\documentclass[12pt,a4paper]{article}
\usepackage[utf8]{inputenc}
\usepackage[russian]{babel}
\usepackage{geometry}
\usepackage{listings}
\usepackage{xcolor}
\usepackage{hyperref}
\usepackage{graphicx}
\usepackage{float}

\geometry{margin=2.5cm}

% Настройка для кода
\lstset{
    language=Python,
    basicstyle=\ttfamily\small,
    keywordstyle=\color{blue}\bfseries,
    commentstyle=\color{green!60!black},
    stringstyle=\color{red},
    numbers=left,
    numberstyle=\tiny\color{gray},
    stepnumber=1,
    numbersep=5pt,
    backgroundcolor=\color{gray!10},
    frame=single,
    breaklines=true,
    breakatwhitespace=true,
    tabsize=4,
    showstringspaces=false
}

\title{Отчет по Django проекту ``Blogicum''}
\author{Студент}
\date{\today}

\begin{document}

\maketitle
\tableofcontents
\newpage

\section{Введение}

Проект \textbf{Blogicum} --- это веб-приложение блога, разработанное на фреймворке Django 3.2. Проект представляет собой полнофункциональную платформу для публикации постов, комментирования и управления пользователями.

\subsection{Технологический стек}
\begin{itemize}
    \item \textbf{Django 5.0+} --- основной веб-фреймворк
    \item \textbf{django-bootstrap5} --- для стилизации интерфейса
    \item \textbf{SQLite} --- база данных
    \item \textbf{Pillow} --- для работы с изображениями
    \item \textbf{pytest} --- для тестирования
\end{itemize}

\section{Структура проекта}

Проект организован по стандартной структуре Django с разделением на приложения:

\begin{verbatim}
blogicum/
├── blogicum/          # Настройки проекта
├── blog/              # Приложение блога
├── pages/             # Статические страницы
├── templates/         # HTML шаблоны
├── static_dev/        # Статические файлы (CSS, JS, изображения)
├── media/             # Загруженные пользователями файлы
└── manage.py          # Утилита управления Django
\end{verbatim}

\section{Описание файлов проекта}

\subsection{Корневой уровень}

\subsubsection{manage.py}
\textbf{Расположение:} \texttt{blogicum/manage.py}

\textbf{Назначение:} Точка входа для выполнения административных команд Django.

\textbf{Содержимое:}
\begin{lstlisting}
#!/usr/bin/env python
"""Django's command-line utility for administrative tasks."""
import os
import sys

def main():
    """Run administrative tasks."""
    os.environ.setdefault('DJANGO_SETTINGS_MODULE', 'blogicum.settings')
    try:
        from django.core.management import execute_from_command_line
    except ImportError as exc:
        raise ImportError(...) from exc
    execute_from_command_line(sys.argv)

if __name__ == '__main__':
    main()
\end{lstlisting}

\textbf{Использование:}
\begin{itemize}
    \item \texttt{python manage.py runserver} --- запуск сервера разработки
    \item \texttt{python manage.py migrate} --- применение миграций БД
    \item \texttt{python manage.py createsuperuser} --- создание администратора
\end{itemize}

\subsubsection{requirements.txt}
\textbf{Расположение:} \texttt{requirements.txt}

\textbf{Назначение:} Список зависимостей проекта.

\textbf{Содержимое:}
\begin{lstlisting}
Django>=5.0
django-bootstrap5>=24.0
pytest>=7.4
pytest-django>=4.5.0
mixer>=7.2.2
Faker>=5.4.0,<12.1
pep8-naming>=0.13.3
flake8>=5.0.4
flake8-docstrings>=1.7.0
beautifulsoup4==4.11.2
Pillow>=10.0.0
\end{lstlisting}

\subsection{Папка blogicum (настройки проекта)}

\subsubsection{settings.py}
\textbf{Расположение:} \texttt{blogicum/blogicum/settings.py}

\textbf{Назначение:} Главный файл конфигурации Django проекта.

\textbf{Ключевые настройки:}

\begin{itemize}
    \item \textbf{BASE\_DIR} --- базовый путь проекта
    \item \textbf{SECRET\_KEY} --- секретный ключ для криптографии
    \item \textbf{DEBUG = True} --- режим отладки (для продакшена должно быть False)
    \item \textbf{INSTALLED\_APPS} --- список установленных приложений:
    \begin{itemize}
        \item Стандартные приложения Django (admin, auth, sessions и т.д.)
        \item \texttt{blog.apps.BlogConfig} --- приложение блога
        \item \texttt{pages.apps.PagesConfig} --- приложение статических страниц
        \item \texttt{django\_bootstrap5} --- для Bootstrap стилей
    \end{itemize}
    \item \textbf{MIDDLEWARE} --- промежуточное ПО для обработки запросов
    \item \textbf{ROOT\_URLCONF = 'blogicum.urls'} --- главный файл маршрутизации
    \item \textbf{DATABASES} --- настройки БД (SQLite по умолчанию)
    \item \textbf{TEMPLATES} --- настройки шаблонов:
    \begin{itemize}
        \item \texttt{TEMPLATES\_DIR = BASE\_DIR / 'templates'} --- путь к шаблонам
    \end{itemize}
    \item \textbf{STATIC\_URL = '/static/'} --- URL для статических файлов
    \item \textbf{STATICFILES\_DIRS = [BASE\_DIR / 'static\_dev']} --- папка со статикой
    \item \textbf{MEDIA\_ROOT = BASE\_DIR / 'media'} --- папка для загруженных файлов
    \item \textbf{MEDIA\_URL = '/media/'} --- URL для медиа-файлов
    \item \textbf{EMAIL\_BACKEND} --- файловый бэкенд для email (для разработки)
    \item \textbf{LOGIN\_URL = '/auth/login/'} --- URL для входа
    \item \textbf{LANGUAGE\_CODE = 'ru-RU'} --- русский язык интерфейса
    \item \textbf{TIME\_ZONE = 'UTC'} --- часовой пояс
\end{itemize}

\subsubsection{urls.py (главный)}
\textbf{Расположение:} \texttt{blogicum/blogicum/urls.py}

\textbf{Назначение:} Главный файл маршрутизации URL проекта.

\textbf{Содержимое:}
\begin{lstlisting}
from django.contrib import admin
from django.urls import include, path
from django.conf import settings
from django.conf.urls.static import static

urlpatterns = [
    path('admin/', admin.site.urls),           # Админ-панель
    path('', include('blog.urls')),            # Маршруты блога
    path('pages/', include('pages.urls')),     # Статические страницы
    path('auth/', include('django.contrib.auth.urls')),  # Авторизация
]

# Подключение медиа-файлов в режиме разработки
if settings.DEBUG:
    urlpatterns += static(
        settings.MEDIA_URL, document_root=settings.MEDIA_ROOT
    )

# Обработчики ошибок
handler403 = 'pages.views.csrf_failure'
handler404 = 'pages.views.page_not_found'
handler500 = 'pages.views.server_error'
\end{lstlisting}

\textbf{Маршруты:}
\begin{itemize}
    \item \texttt{/admin/} --- административная панель Django
    \item \texttt{/} --- главная страница (обрабатывается blog.urls)
    \item \texttt{/pages/} --- статические страницы
    \item \texttt{/auth/} --- авторизация (встроенная в Django)
\end{itemize}

\subsubsection{wsgi.py}
\textbf{Расположение:} \texttt{blogicum/blogicum/wsgi.py}

\textbf{Назначение:} WSGI конфигурация для развертывания на продакшн-сервере.

\textbf{Содержимое:}
\begin{lstlisting}
import os
from django.core.wsgi import get_wsgi_application

os.environ.setdefault('DJANGO_SETTINGS_MODULE', 'blogicum.settings')
application = get_wsgi_application()
\end{lstlisting}

\textbf{Использование:} Используется веб-серверами (Apache, Nginx) для запуска Django приложения.

\subsubsection{asgi.py}
\textbf{Расположение:} \texttt{blogicum/blogicum/asgi.py}

\textbf{Назначение:} ASGI конфигурация для асинхронных веб-серверов.

\textbf{Содержимое:} Аналогичен wsgi.py, но для ASGI протокола.

\subsection{Приложение blog}

\subsubsection{models.py}
\textbf{Расположение:} \texttt{blogicum/blog/models.py}

\textbf{Назначение:} Определение моделей данных (структура БД).

\textbf{Модели:}

\begin{enumerate}
    \item \textbf{Category} (Категории постов)
    \begin{itemize}
        \item \texttt{title} --- название категории (CharField, max\_length=256)
        \item \texttt{description} --- описание (TextField)
        \item \texttt{slug} --- URL-идентификатор (SlugField, unique=True)
        \item \texttt{is\_published} --- флаг публикации (BooleanField, default=True)
        \item \texttt{created\_at} --- дата создания (DateTimeField, auto\_now\_add=True)
    \end{itemize}
    
    \item \textbf{Location} (Местоположения)
    \begin{itemize}
        \item \texttt{name} --- название места (CharField, max\_length=256)
        \item \texttt{is\_published} --- флаг публикации
        \item \texttt{created\_at} --- дата создания
    \end{itemize}
    
    \item \textbf{Post} (Публикации) --- основная модель
    \begin{itemize}
        \item \texttt{title} --- заголовок поста (CharField, max\_length=256)
        \item \texttt{text} --- текст поста (TextField)
        \item \texttt{pub\_date} --- дата публикации (DateTimeField)
        \item \texttt{author} --- автор (ForeignKey к User)
        \item \texttt{location} --- местоположение (ForeignKey к Location, null=True)
        \item \texttt{category} --- категория (ForeignKey к Category, обязательна)
        \item \texttt{is\_published} --- флаг публикации
        \item \texttt{image} --- изображение (ImageField, upload\_to='posts')
        \item \texttt{created\_at} --- дата создания
        \item \textbf{Meta:} ordering = ('-pub\_date',) --- сортировка по дате (новые первыми)
    \end{itemize}
    
    \item \textbf{Comment} (Комментарии)
    \begin{itemize}
        \item \texttt{text} --- текст комментария (TextField)
        \item \texttt{post} --- связанный пост (ForeignKey к Post, related\_name='comments')
        \item \texttt{author} --- автор комментария (ForeignKey к User)
        \item \texttt{created\_at} --- дата создания
        \item \textbf{Meta:} ordering = ('created\_at',) --- сортировка по дате
    \end{itemize}
\end{enumerate}

\subsubsection{views.py}
\textbf{Расположение:} \texttt{blogicum/blog/views.py}

\textbf{Назначение:} Обработчики HTTP-запросов (контроллеры).

\textbf{Функции-представления:}

\begin{enumerate}
    \item \textbf{index(request)} --- главная страница
    \begin{itemize}
        \item Получает опубликованные посты
        \item Фильтрует: \texttt{pub\_date <= now}, \texttt{is\_published=True}
        \item Добавляет количество комментариев через \texttt{annotate(comment\_count=Count('comments'))}
        \item Использует пагинацию (10 постов на страницу)
        \item Шаблон: \texttt{blog/index.html}
    \end{itemize}
    
    \item \textbf{post\_detail(request, id)} --- детальная страница поста
    \begin{itemize}
        \item Получает пост по ID
        \item Проверяет права доступа (неопубликованные посты видны только автору)
        \item Проверяет дату публикации (будущие посты недоступны)
        \item Передает форму комментария и список комментариев
        \item Шаблон: \texttt{blog/detail.html}
    \end{itemize}
    
    \item \textbf{category\_posts(request, category\_slug)} --- посты категории
    \begin{itemize}
        \item Фильтрует посты по категории (по slug)
        \item Использует пагинацию
        \item Шаблон: \texttt{blog/category.html}
    \end{itemize}
    
    \item \textbf{profile(request, username)} --- профиль пользователя
    \begin{itemize}
        \item Показывает все посты пользователя
        \item Использует пагинацию
        \item Шаблон: \texttt{blog/profile.html}
    \end{itemize}
    
    \item \textbf{edit\_profile(request)} --- редактирование профиля
    \begin{itemize}
        \item Декоратор: \texttt{@login\_required} --- требует авторизации
        \item Использует форму \texttt{UserForm}
        \item Шаблон: \texttt{blog/user.html}
    \end{itemize}
    
    \item \textbf{create\_post(request)} --- создание поста
    \begin{itemize}
        \item Декоратор: \texttt{@login\_required}
        \item Обрабатывает POST-запрос с формой \texttt{PostForm}
        \item Автоматически устанавливает автора: \texttt{post.author = request.user}
        \item Шаблон: \texttt{blog/create.html}
    \end{itemize}
    
    \item \textbf{edit\_post(request, post\_id)} --- редактирование поста
    \begin{itemize}
        \item Декоратор: \texttt{@login\_required}
        \item Проверяет, что пользователь --- автор поста
        \item Шаблон: \texttt{blog/create.html}
    \end{itemize}
    
    \item \textbf{delete\_post(request, post\_id)} --- удаление поста
    \begin{itemize}
        \item Декоратор: \texttt{@login\_required}
        \item Проверяет права автора
        \item Удаляет пост при POST-запросе
    \end{itemize}
    
    \item \textbf{add\_comment(request, post\_id)} --- добавление комментария
    \begin{itemize}
        \item Декоратор: \texttt{@login\_required}
        \item Создает комментарий и связывает с постом и автором
    \end{itemize}
    
    \item \textbf{edit\_comment(request, post\_id, comment\_id)} --- редактирование комментария
    \begin{itemize}
        \item Декоратор: \texttt{@login\_required}
        \item Проверяет права автора комментария
        \item Шаблон: \texttt{blog/comment.html}
    \end{itemize}
    
    \item \textbf{delete\_comment(request, post\_id, comment\_id)} --- удаление комментария
    \begin{itemize}
        \item Декоратор: \texttt{@login\_required}
        \item Проверяет права автора
    \end{itemize}
    
    \item \textbf{registration(request)} --- регистрация пользователя
    \begin{itemize}
        \item Использует встроенную форму \texttt{UserCreationForm}
        \item Шаблон: \texttt{registration/registration\_form.html}
    \end{itemize}
\end{enumerate}

\subsubsection{forms.py}
\textbf{Расположение:} \texttt{blogicum/blog/forms.py}

\textbf{Назначение:} Определение форм для работы с данными.

\textbf{Формы:}

\begin{enumerate}
    \item \textbf{PostForm} --- форма для создания/редактирования поста
    \begin{itemize}
        \item Поля: \texttt{title}, \texttt{text}, \texttt{pub\_date}, \texttt{location}, \texttt{category}, \texttt{image}
        \item Использует Bootstrap классы для стилизации (\texttt{form-control})
        \item \texttt{pub\_date} --- виджет \texttt{datetime-local} для выбора даты и времени
    \end{itemize}
    
    \item \textbf{CommentForm} --- форма для комментариев
    \begin{itemize}
        \item Поле: \texttt{text} (Textarea, 5 строк)
    \end{itemize}
    
    \item \textbf{UserForm} --- форма для редактирования профиля
    \begin{itemize}
        \item Поля: \texttt{first\_name}, \texttt{last\_name}, \texttt{username}, \texttt{email}
    \end{itemize}
\end{enumerate}

\subsubsection{urls.py}
\textbf{Расположение:} \texttt{blogicum/blog/urls.py}

\textbf{Назначение:} Маршрутизация URL для приложения blog.

\textbf{Маршруты:}
\begin{lstlisting}
app_name = 'blog'

urlpatterns = [
    path('', views.index, name='index'),                    # Главная
    path('posts/<int:id>/', views.post_detail, ...),        # Детали поста
    path('category/<slug:category_slug>/', ...),            # Категория
    path('profile/<str:username>/', views.profile, ...),    # Профиль
    path('edit_profile/', views.edit_profile, ...),         # Редактирование профиля
    path('posts/create/', views.create_post, ...),         # Создание поста
    path('posts/<int:post_id>/edit/', views.edit_post, ...), # Редактирование
    path('posts/<int:post_id>/delete/', views.delete_post, ...), # Удаление
    path('posts/<int:post_id>/comment/', views.add_comment, ...), # Комментарий
    path('posts/<int:post_id>/edit_comment/<int:comment_id>/', ...), # Редактирование комментария
    path('posts/<int:post_id>/delete_comment/<int:comment_id>/', ...), # Удаление комментария
    path('auth/registration/', views.registration, ...),   # Регистрация
]
\end{lstlisting}

\subsubsection{admin.py}
\textbf{Расположение:} \texttt{blogicum/blog/admin.py}

\textbf{Назначение:} Настройка административной панели Django.

\textbf{Админ-классы:}

\begin{enumerate}
    \item \textbf{CategoryAdmin} --- управление категориями
    \begin{itemize}
        \item Отображает: \texttt{title}, \texttt{is\_published}, \texttt{created\_at}
        \item Редактируемое поле: \texttt{is\_published}
        \item Поиск: по \texttt{title}
        \item Фильтры: по \texttt{is\_published}, \texttt{created\_at}
    \end{itemize}
    
    \item \textbf{LocationAdmin} --- управление местоположениями
    \begin{itemize}
        \item Аналогично CategoryAdmin
    \end{itemize}
    
    \item \textbf{PostAdmin} --- управление постами
    \begin{itemize}
        \item Отображает: \texttt{title}, \texttt{author}, \texttt{category}, \texttt{location}, \texttt{pub\_date}, \texttt{is\_published}, \texttt{created\_at}
        \item Редактируемое поле: \texttt{is\_published}
        \item Поиск: по \texttt{title}, \texttt{text}
        \item Фильтры: по \texttt{is\_published}, \texttt{category}, \texttt{location}, \texttt{pub\_date}
        \item Иерархия дат: \texttt{date\_hierarchy = 'pub\_date'}
    \end{itemize}
    
    \item \textbf{CommentAdmin} --- управление комментариями
    \begin{itemize}
        \item Отображает: \texttt{text}, \texttt{post}, \texttt{author}, \texttt{created\_at}
        \item Поиск: по \texttt{text}
        \item Фильтры: по \texttt{created\_at}
    \end{itemize}
\end{enumerate}

\subsubsection{apps.py}
\textbf{Расположение:} \texttt{blogicum/blog/apps.py}

\textbf{Назначение:} Конфигурация приложения blog.

\textbf{Содержимое:}
\begin{lstlisting}
class BlogConfig(AppConfig):
    default_auto_field = 'django.db.models.BigAutoField'
    name = 'blog'
    verbose_name = 'Блог'
\end{lstlisting}

\subsection{Приложение pages}

\subsubsection{views.py}
\textbf{Расположение:} \texttt{blogicum/pages/views.py}

\textbf{Назначение:} Обработчики для статических страниц и ошибок.

\textbf{Функции:}

\begin{enumerate}
    \item \textbf{AboutView} --- страница ``О проекте''
    \begin{itemize}
        \item Класс-представление: \texttt{TemplateView}
        \item Шаблон: \texttt{pages/about.html}
    \end{itemize}
    
    \item \textbf{RulesView} --- страница ``Правила''
    \begin{itemize}
        \item Шаблон: \texttt{pages/rules.html}
    \end{itemize}
    
    \item \textbf{csrf\_failure(request, reason='')} --- обработка ошибки CSRF
    \begin{itemize}
        \item Возвращает шаблон \texttt{pages/403csrf.html} со статусом 403
    \end{itemize}
    
    \item \textbf{page\_not\_found(request, exception)} --- обработка 404
    \begin{itemize}
        \item Возвращает шаблон \texttt{pages/404.html} со статусом 404
    \end{itemize}
    
    \item \textbf{server\_error(request)} --- обработка 500
    \begin{itemize}
        \item Возвращает шаблон \texttt{pages/500.html} со статусом 500
    \end{itemize}
\end{enumerate}

\subsubsection{urls.py}
\textbf{Расположение:} \texttt{blogicum/pages/urls.py}

\textbf{Содержимое:}
\begin{lstlisting}
app_name = 'pages'

urlpatterns = [
    path('about/', views.AboutView.as_view(), name='about'),
    path('rules/', views.RulesView.as_view(), name='rules'),
]
\end{lstlisting}

\section{Шаблоны (Templates)}

\textbf{Расположение:} \texttt{blogicum/templates/}

Шаблоны организованы по приложениям и используют наследование от базового шаблона.

\subsection{Базовый шаблон}

\subsubsection{base.html}
\textbf{Расположение:} \texttt{templates/base.html}

\textbf{Назначение:} Базовый шаблон для всех страниц.

\textbf{Структура:}
\begin{itemize}
    \item Загружает статические файлы и Bootstrap5
    \item Подключает favicon и мета-теги
    \item Включает шаблоны: \texttt{includes/header.html} и \texttt{includes/footer.html}
    \item Блок \texttt{title} --- для заголовка страницы
    \item Блок \texttt{content} --- для основного содержимого
\end{itemize}

\subsection{Шаблоны блога}

\textbf{Расположение:} \texttt{templates/blog/}

\begin{enumerate}
    \item \textbf{index.html} --- главная страница
    \begin{itemize}
        \item Наследуется от \texttt{base.html}
        \item Цикл по \texttt{page\_obj} (посты)
        \item Включает \texttt{includes/post\_card.html} для каждого поста
        \item Включает \texttt{includes/paginator.html} для пагинации
    \end{itemize}
    
    \item \textbf{detail.html} --- детальная страница поста
    \begin{itemize}
        \item Отображает полный текст поста
        \item Показывает форму комментария
        \item Выводит список комментариев через \texttt{includes/comments.html}
    \end{itemize}
    
    \item \textbf{category.html} --- страница категории
    \begin{itemize}
        \item Аналогична index.html, но фильтрует по категории
    \end{itemize}
    
    \item \textbf{profile.html} --- профиль пользователя
    \begin{itemize}
        \item Показывает информацию о пользователе
        \item Список постов пользователя с пагинацией
    \end{itemize}
    
    \item \textbf{create.html} --- создание/редактирование поста
    \begin{itemize}
        \item Форма \texttt{PostForm} с Bootstrap стилями
        \item Кнопки ``Сохранить'' и ``Удалить'' (для редактирования)
    \end{itemize}
    
    \item \textbf{user.html} --- редактирование профиля
    \begin{itemize}
        \item Форма \texttt{UserForm}
    \end{itemize}
    
    \item \textbf{comment.html} --- редактирование/удаление комментария
    \begin{itemize}
        \item Форма для редактирования или подтверждение удаления
    \end{itemize}
\end{enumerate}

\subsection{Включаемые шаблоны}

\textbf{Расположение:} \texttt{templates/includes/}

\begin{enumerate}
    \item \textbf{header.html} --- шапка сайта
    \begin{itemize}
        \item Навигационное меню с Bootstrap
        \item Логотип ``Блогикум''
        \item Ссылки: ``О проекте'', ``Правила''
        \item Условный блок: если пользователь авторизован --- ``Написать пост'', профиль, ``Выйти''
        \item Если не авторизован --- ``Войти'', ``Регистрация''
    \end{itemize}
    
    \item \textbf{footer.html} --- подвал сайта
    \begin{itemize}
        \item Копирайт и информация о проекте
    \end{itemize}
    
    \item \textbf{post\_card.html} --- карточка поста
    \begin{itemize}
        \item Отображает заголовок, текст (первые 10 слов), дату, автора, категорию, местоположение
        \item Ссылка ``Читать полный текст''
        \item Если есть изображение --- отображает его
    \end{itemize}
    
    \item \textbf{comments.html} --- список комментариев
    \begin{itemize}
        \item Цикл по комментариям
        \item Для каждого комментария: автор, дата, текст
        \item Кнопки редактирования/удаления (только для автора)
    \end{itemize}
    
    \item \textbf{paginator.html} --- пагинация
    \begin{itemize}
        \item Навигация по страницам (``Предыдущая'', ``Следующая'')
        \item Номера страниц
    \end{itemize}
    
    \item \textbf{category\_link.html} --- ссылка на категорию
    \begin{itemize}
        \item Используется в карточках постов
    \end{itemize}
\end{enumerate}

\subsection{Страницы ошибок}

\textbf{Расположение:} \texttt{templates/pages/}

\begin{enumerate}
    \item \textbf{403csrf.html} --- ошибка CSRF защиты
    \item \textbf{404.html} --- страница не найдена
    \item \textbf{500.html} --- внутренняя ошибка сервера
    \item \textbf{about.html} --- о проекте
    \item \textbf{rules.html} --- правила использования
\end{enumerate}

\subsection{Шаблоны авторизации}

\textbf{Расположение:} \texttt{templates/registration/}

Используются встроенные шаблоны Django для авторизации:
\begin{itemize}
    \item \texttt{login.html} --- форма входа
    \item \texttt{registration\_form.html} --- форма регистрации
    \item \texttt{password\_change\_form.html} --- смена пароля
    \item \texttt{password\_reset\_form.html} --- сброс пароля
    \item И другие шаблоны для восстановления пароля
\end{itemize}

\section{Статические файлы}

\textbf{Расположение:} \texttt{blogicum/static\_dev/}

\begin{itemize}
    \item \textbf{css/bootstrap.min.css} --- стили Bootstrap
    \item \textbf{img/logo.png} --- логотип сайта
    \item \textbf{img/fav/} --- favicon файлы для разных устройств
\end{itemize}

\section{Медиа-файлы}

\textbf{Расположение:} \texttt{blogicum/media/}

\begin{itemize}
    \item \textbf{posts/} --- загруженные изображения постов
    \item Файлы сохраняются автоматически при загрузке через форму
\end{itemize}

\section{База данных}

\textbf{Файл:} \texttt{blogicum/db.sqlite3}

SQLite база данных содержит таблицы:
\begin{itemize}
    \item \texttt{blog\_category} --- категории
    \item \texttt{blog\_location} --- местоположения
    \item \texttt{blog\_post} --- посты
    \item \texttt{blog\_comment} --- комментарии
    \item \texttt{auth\_user} --- пользователи (встроенная модель Django)
    \item И другие служебные таблицы Django
\end{itemize}

\section{Миграции}

\textbf{Расположение:} \texttt{blogicum/blog/migrations/}

\begin{itemize}
    \item \texttt{0001\_initial.py} --- первичная миграция (создание таблиц)
    \item \texttt{0002\_alter\_post\_author\_alter\_post\_category\_and\_more.py} --- изменения полей
    \item \texttt{0003\_alter\_post\_options\_post\_image\_comment.py} --- добавление изображений и комментариев
\end{itemize}

\section{Тестирование}

\textbf{Расположение:} \texttt{tests/}

Проект использует \textbf{pytest} и \textbf{pytest-django} для автоматизированного тестирования. Все тесты находятся в папке \texttt{tests/}.

\subsection{Конфигурация тестирования}

\subsubsection{pytest.ini}
\textbf{Расположение:} \texttt{pytest.ini}

\textbf{Содержимое:}
\begin{lstlisting}
[pytest]
pythonpath = blogicum/ .
DJANGO_SETTINGS_MODULE = blogicum.settings
norecursedirs = env/*
addopts = -rE -vv --show-capture=no --disable-warnings
testpaths = tests/
python_files = test_*.py
django_debug_mode = true
\end{lstlisting}

\textbf{Настройки:}
\begin{itemize}
    \item \texttt{pythonpath} --- пути для импорта модулей
    \item \texttt{DJANGO\_SETTINGS\_MODULE} --- модуль настроек Django
    \item \texttt{testpaths = tests/} --- папка с тестами
    \item \texttt{python\_files = test\_*.py} --- паттерн имен тестовых файлов
\end{itemize}

\subsection{Запуск тестов}

\textbf{Команда для запуска всех тестов:}
\begin{lstlisting}
pytest
\end{lstlisting}

\textbf{Запуск конкретного теста:}
\begin{lstlisting}
pytest tests/test_post.py
pytest tests/test_comment.py::test_comment
\end{lstlisting}

\textbf{Запуск с подробным выводом:}
\begin{lstlisting}
pytest -v
\end{lstlisting}

\subsection{Структура тестов}

Тесты организованы по функциональным областям:

\begin{itemize}
    \item \texttt{test\_post.py} --- тесты функциональности постов
    \item \texttt{test\_comment.py} --- тесты комментариев
    \item \texttt{test\_users.py} --- тесты пользователей и профилей
    \item \texttt{test\_content.py} --- тесты отображения контента
    \item \texttt{test\_static\_pages.py} --- тесты статических страниц
    \item \texttt{test\_edit.py} --- вспомогательные функции для тестирования редактирования
    \item \texttt{test\_err\_pages.py} --- тесты страниц ошибок (404, 500, CSRF)
    \item \texttt{test\_emails.py} --- тесты отправки email
\end{itemize}

\subsection{Вспомогательные компоненты}

\subsubsection{conftest.py}
\textbf{Расположение:} \texttt{tests/conftest.py}

\textbf{Назначение:} Глобальные фикстуры и вспомогательные функции для всех тестов.

\textbf{Ключевые фикстуры:}

\begin{enumerate}
    \item \textbf{user} --- создает тестового пользователя
    \begin{lstlisting}
@pytest.fixture
def user(mixer):
    User = get_user_model()
    user = mixer.blend(User)
    return user
    \end{lstlisting}
    
    \item \textbf{user\_client} --- клиент с авторизованным пользователем
    \begin{lstlisting}
@pytest.fixture
def user_client(user):
    client = Client()
    client.force_login(user)
    return client
    \end{lstlisting}
    
    \item \textbf{unlogged\_client} --- неавторизованный клиент
    \item \textbf{another\_user} --- второй пользователь для проверки прав доступа
    \item \textbf{another\_user\_client} --- клиент второго пользователя
\end{enumerate}

\textbf{Вспомогательные функции:}
\begin{itemize}
    \item \texttt{get\_get\_response\_safely()} --- безопасное получение HTTP-ответа
    \item \texttt{get\_a\_post\_get\_response\_safely()} --- получение страницы поста
    \item \texttt{get\_create\_a\_post\_get\_response\_safely()} --- получение страницы создания поста
    \item \texttt{\_TestModelAttrs} --- базовый класс для тестирования атрибутов моделей
\end{itemize}

\subsubsection{Фикстуры (fixtures)}
\textbf{Расположение:} \texttt{tests/fixtures/}

Создают тестовые данные:
\begin{itemize}
    \item \texttt{posts.py} --- фикстуры для постов
    \item \texttt{comments.py} --- фикстуры для комментариев
    \item \texttt{categories.py} --- фикстуры для категорий
    \item \texttt{locations.py} --- фикстуры для местоположений
\end{itemize}

\subsubsection{Адаптеры (adapters)}
\textbf{Расположение:} \texttt{tests/adapters/}

Классы-обертки для работы с моделями в тестах:
\begin{itemize}
    \item \texttt{post.py} --- \texttt{PostModelAdapter}
    \item \texttt{comment.py} --- \texttt{CommentModelAdapter}
    \item \texttt{user.py} --- \texttt{UserModelAdapter}
    \item \texttt{model\_adapter.py} --- базовый адаптер
\end{itemize}

\subsubsection{Тестеры форм (form)}
\textbf{Расположение:} \texttt{tests/form/}

Классы для тестирования форм:
\begin{itemize}
    \item \texttt{base\_form\_tester.py} --- базовый класс для тестирования форм
    \item \texttt{post/create\_form\_tester.py} --- тестирование создания поста
    \item \texttt{post/edit\_form\_tester.py} --- тестирование редактирования поста
    \item \texttt{post/delete\_tester.py} --- тестирование удаления поста
    \item \texttt{comment/create\_form\_tester.py} --- тестирование создания комментария
    \item \texttt{comment/edit\_form\_tester.py} --- тестирование редактирования комментария
    \item \texttt{comment/delete\_tester.py} --- тестирование удаления комментария
    \item \texttt{user/edit\_form\_tester.py} --- тестирование редактирования профиля
\end{itemize}

\subsection{Описание тестов}

\subsubsection{test\_post.py}
\textbf{Расположение:} \texttt{tests/test\_post.py}

\textbf{Что проверяет:}

\begin{enumerate}
    \item \textbf{Тесты модели Post}
    \begin{itemize}
        \item Проверка наличия полей: \texttt{image} (ImageField), \texttt{pub\_date} (DateTimeField)
        \item Проверка параметров полей (например, \texttt{auto\_now\_add=False} для \texttt{pub\_date})
        \item Проверка автоматического заполнения \texttt{created\_at} при создании
    \end{itemize}
    
    \item \textbf{Тесты создания поста}
    \begin{itemize}
        \item Неавторизованный пользователь не может создать пост
        \item Авторизованный пользователь может создать пост
        \item После создания происходит редирект на страницу профиля
        \item Изображения отображаются на странице поста
        \item Валидация формы создания поста
    \end{itemize}
    
    \item \textbf{Тесты редактирования поста}
    \begin{itemize}
        \item Только автор может редактировать свой пост
        \item Другие пользователи не могут редактировать чужой пост
        \item Неавторизованные пользователи не могут редактировать
        \item Проверка URL: \texttt{/posts/<post\_id>/edit/}
    \end{itemize}
    
    \item \textbf{Тесты удаления поста}
    \begin{itemize}
        \item Только автор может удалить свой пост
        \item После удаления пост исчезает из БД
        \item При обращении к удаленному посту возвращается 404
        \item При обращении к несуществующему посту возвращается 404
    \end{itemize}
    
    \item \textbf{Тесты доступа к постам}
    \begin{itemize}
        \item Неопубликованные посты видны только автору
        \item Посты с неопубликованной категорией видны только автору
        \item Отложенные посты (с будущей датой) видны только автору
        \item Опубликованные посты видны всем
    \end{itemize}
\end{enumerate}

\textbf{Где реализовано:}
\begin{itemize}
    \item Создание: \texttt{blog/views.py::create\_post()}
    \item Редактирование: \texttt{blog/views.py::edit\_post()}
    \item Удаление: \texttt{blog/views.py::delete\_post()}
    \item Детали: \texttt{blog/views.py::post\_detail()}
    \item Модель: \texttt{blog/models.py::Post}
\end{itemize}

\subsubsection{test\_comment.py}
\textbf{Расположение:} \texttt{tests/test\_comment.py}

\textbf{Что проверяет:}

\begin{enumerate}
    \item \textbf{Тесты модели Comment}
    \begin{itemize}
        \item Наличие полей: \texttt{post} (ForeignKey), \texttt{author} (ForeignKey), \texttt{text} (TextField)
        \item Поле \texttt{created\_at} с \texttt{auto\_now\_add=True}
        \item Автоматическое заполнение \texttt{created\_at} при создании
    \end{itemize}
    
    \item \textbf{Тесты создания комментария}
    \begin{itemize}
        \item Неавторизованный пользователь не может создать комментарий
        \item Авторизованный пользователь может создать комментарий
        \item Комментарий связывается с постом и автором
        \item После создания происходит редирект на страницу поста
        \item Комментарии отображаются на странице поста
    \end{itemize}
    
    \item \textbf{Тесты редактирования комментария}
    \begin{itemize}
        \item Только автор может редактировать свой комментарий
        \item Проверка URL: \texttt{/posts/<post\_id>/edit\_comment/<comment\_id>/}
    \end{itemize}
    
    \item \textbf{Тесты удаления комментария}
    \begin{itemize}
        \item Только автор может удалить свой комментарий
        \item При обращении к удаленному комментарию возвращается 404
        \item При обращении к комментарию несуществующего поста возвращается 404
    \end{itemize}
    
    \item \textbf{Тесты отображения количества комментариев}
    \begin{itemize}
        \item На главной странице отображается количество комментариев в формате \texttt{(N)}
        \item На странице профиля отображается количество комментариев
    \end{itemize}
\end{enumerate}

\textbf{Где реализовано:}
\begin{itemize}
    \item Создание: \texttt{blog/views.py::add\_comment()}
    \item Редактирование: \texttt{blog/views.py::edit\_comment()}
    \item Удаление: \texttt{blog/views.py::delete\_comment()}
    \item Модель: \texttt{blog/models.py::Comment}
    \item Отображение: \texttt{blog/views.py::index()} (через \texttt{annotate(comment\_count=Count('comments'))})
\end{itemize}

\subsubsection{test\_users.py}
\textbf{Расположение:} \texttt{tests/test\_users.py}

\textbf{Что проверяет:}

\begin{enumerate}
    \item \textbf{Тесты страницы профиля}
    \begin{itemize}
        \item При обращении к несуществующему пользователю возвращается 404
        \item На странице профиля отображается имя и фамилия пользователя
        \item Страница профиля доступна всем (авторизованным и неавторизованным)
    \end{itemize}
    
    \item \textbf{Тесты редактирования профиля}
    \begin{itemize}
        \item Ссылки ``Редактировать профиль'' и ``Изменить пароль'' видны только владельцу профиля
        \item Неавторизованные пользователи не видят ссылки редактирования
        \item Другие пользователи не видят ссылки редактирования
        \item Проверка URL изменения пароля: \texttt{/auth/password\_change/}
    \end{itemize}
    
    \item \textbf{Тесты маршрутизации}
    \begin{itemize}
        \item Подключены маршруты из \texttt{django.contrib.auth.urls}
        \item Переопределен маршрут \texttt{auth/registration/}
        \item Проверка наличия всех шаблонов авторизации
    \end{itemize}
\end{enumerate}

\textbf{Где реализовано:}
\begin{itemize}
    \item Профиль: \texttt{blog/views.py::profile()}
    \item Редактирование: \texttt{blog/views.py::edit\_profile()}
    \item Регистрация: \texttt{blog/views.py::registration()}
    \item Маршруты: \texttt{blogicum/urls.py} и \texttt{blog/urls.py}
    \item Шаблоны: \texttt{templates/registration/*.html}
\end{itemize}

\subsubsection{test\_content.py}
\textbf{Расположение:} \texttt{tests/test\_content.py}

\textbf{Назначение:} Тестирование отображения контента на страницах.

\textbf{Что проверяет:}
\begin{itemize}
    \item Корректное отображение постов на главной странице
    \item Отображение постов на странице категории
    \item Отображение постов на странице профиля
    \item Пагинация работает корректно
    \item Использование правильных шаблонов
\end{itemize}

\textbf{Где реализовано:}
\begin{itemize}
    \item Главная: \texttt{blog/views.py::index()}
    \item Категория: \texttt{blog/views.py::category\_posts()}
    \item Профиль: \texttt{blog/views.py::profile()}
    \item Шаблоны: \texttt{templates/blog/index.html}, \texttt{templates/blog/category.html}, \texttt{templates/blog/profile.html}
\end{itemize}

\subsubsection{test\_static\_pages.py}
\textbf{Расположение:} \texttt{tests/test\_static\_pages.py}

\textbf{Что проверяет:}
\begin{itemize}
    \item Статические страницы используют CBV (Class-Based Views)
    \item В \texttt{pages/urls.py} определен \texttt{app\_name}
    \item Маршруты подключены через \texttt{TemplateView}
\end{itemize}

\textbf{Где реализовано:}
\begin{itemize}
    \item Представления: \texttt{pages/views.py::AboutView}, \texttt{pages/views.py::RulesView}
    \item Маршруты: \texttt{pages/urls.py}
\end{itemize}

\subsubsection{test\_err\_pages.py}
\textbf{Расположение:} \texttt{tests/test\_err\_pages.py}

\textbf{Что проверяет:}
\begin{itemize}
    \item Обработчик 404 возвращает правильный шаблон
    \item Обработчик 500 возвращает правильный шаблон
    \item Обработчик CSRF (403) возвращает правильный шаблон
\end{itemize}

\textbf{Где реализовано:}
\begin{itemize}
    \item Обработчики: \texttt{pages/views.py::page\_not\_found()}, \texttt{pages/views.py::server\_error()}, \texttt{pages/views.py::csrf\_failure()}
    \item Настройка: \texttt{blogicum/urls.py} (handler403, handler404, handler500)
    \item Шаблоны: \texttt{templates/pages/404.html}, \texttt{templates/pages/500.html}, \texttt{templates/pages/403csrf.html}
\end{itemize}

\subsection{Типы проверок в тестах}

\subsubsection{Проверка моделей}
\begin{itemize}
    \item Наличие полей с правильными типами
    \item Параметры полей (например, \texttt{auto\_now\_add}, \texttt{max\_length})
    \item Связи между моделями (ForeignKey)
    \item Автоматическое заполнение полей при создании
\end{itemize}

\subsubsection{Проверка прав доступа}
\begin{itemize}
    \item Неавторизованные пользователи не могут создавать/редактировать/удалять
    \item Только автор может редактировать/удалять свои объекты
    \item Другие пользователи не могут изменять чужие объекты
    \item Проверка доступа к неопубликованному контенту
\end{itemize}

\subsubsection{Проверка HTTP-ответов}
\begin{itemize}
    \item Статус коды (200, 404, 403)
    \item Редиректы после создания/редактирования
    \item Корректность URL-маршрутов
\end{itemize}

\subsubsection{Проверка форм}
\begin{itemize}
    \item Валидация данных
    \item Наличие всех необходимых полей
    \item Метод отправки (POST)
    \item Обработка файлов (изображения)
\end{itemize}

\subsubsection{Проверка контента}
\begin{itemize}
    \item Отображение данных на страницах
    \item Количество элементов на странице
    \item Пагинация
    \item Наличие изображений
\end{itemize}

\subsection{Инструменты тестирования}

\begin{itemize}
    \item \textbf{pytest} --- фреймворк для тестирования
    \item \textbf{pytest-django} --- интеграция pytest с Django
    \item \textbf{mixer} --- генерация тестовых данных
    \item \textbf{Faker} --- генерация реалистичных тестовых данных
    \item \textbf{BeautifulSoup4} --- парсинг HTML для проверки контента
    \item \textbf{django.test.Client} --- тестовый HTTP-клиент Django
\end{itemize}

\subsection{Пример теста}

Пример полного теста создания поста:

\begin{lstlisting}
@pytest.mark.django_db(transaction=True)
def test_post(...):
    # 1. Получение страницы создания
    create_response = get_create_a_post_get_response_safely(user_client)
    
    # 2. Создание поста через форму
    creation_tester = CreatePostFormTester(...)
    response_on_created, created_items = creation_tester.test_create_several(...)
    
    # 3. Проверка редиректа на профиль
    assert redirected_to_profile
    
    # 4. Проверка отображения изображений
    assert img_count >= expected_img_count
    
    # 5. Тестирование редактирования
    edit_response = _test_edit_post(...)
    
    # 6. Тестирование удаления
    delete_tester.test_delete_item(...)
    
    # 7. Проверка доступа к неопубликованным постам
    check_post_access(...)
\end{lstlisting}

\subsection{Покрытие тестами}

Тесты покрывают:
\begin{itemize}
    \item Все модели (Category, Location, Post, Comment)
    \item Все представления (views)
    \item Все формы (PostForm, CommentForm, UserForm)
    \item Маршрутизацию (URL patterns)
    \item Права доступа
    \item Обработку ошибок
    \item Отображение контента
\end{itemize}

\section{Подробное описание всех 25 тестов}

В проекте реализовано 25 тестов, которые проверяют все аспекты функциональности приложения. Ниже приведено подробное описание каждого теста и указано, где в коде реализована проверяемая функциональность.

\subsection{Тесты модели Post (test\_post.py)}

\subsubsection{Тест 1-2: TestPostModelAttrs (параметризованный класс)}
\textbf{Файл:} \texttt{tests/test\_post.py}, строки 35-58

\textbf{Что проверяет:}
\begin{enumerate}
    \item \textbf{Поле \texttt{image}} --- проверяет наличие поля типа \texttt{ImageField} в модели Post
    \item \textbf{Поле \texttt{pub\_date}} --- проверяет наличие поля типа \texttt{DateTimeField} с параметром \texttt{auto\_now\_add=False}
\end{enumerate}

\textbf{Как работает:}
\begin{itemize}
    \item Использует параметризацию pytest для проверки каждого поля отдельно
    \item Проверяет наличие атрибута в модели
    \item Проверяет тип поля
    \item Проверяет параметры поля (например, \texttt{auto\_now\_add=False} для \texttt{pub\_date})
\end{itemize}

\textbf{Где реализовано:}
\begin{itemize}
    \item Модель: \texttt{blog/models.py::Post}, строки 62-115
    \item Поле \texttt{image}: строка 103-107
    \item Поле \texttt{pub\_date}: строка 68-74
\end{itemize}

\subsubsection{Тест 3: test\_post\_created\_at}
\textbf{Файл:} \texttt{tests/test\_post.py}, строки 60-69

\textbf{Что проверяет:} Автоматическое заполнение поля \texttt{created\_at} текущей датой и временем при создании поста.

\textbf{Как работает:}
\begin{itemize}
    \item Создает пост через фикстуру \texttt{post\_with\_published\_location}
    \item Проверяет, что \texttt{created\_at} установлен в текущее время (с точностью до 1 секунды)
\end{itemize}

\textbf{Где реализовано:}
\begin{itemize}
    \item Модель: \texttt{blog/models.py::Post}, строка 99-102
    \item Параметр \texttt{auto\_now\_add=True} автоматически устанавливает дату при создании
\end{itemize}

\subsubsection{Тест 4: test\_post (комплексный тест)}
\textbf{Файл:} \texttt{tests/test\_post.py}, строки 72-254

\textbf{Что проверяет:} Полный цикл работы с постами --- создание, редактирование, удаление, права доступа.

\textbf{Как работает:}
\begin{enumerate}
    \item \textbf{Создание постов:}
    \begin{itemize}
        \item Проверяет, что неавторизованный пользователь не может создать пост
        \item Проверяет создание нескольких постов авторизованным пользователем
        \item Проверяет редирект на страницу профиля после создания
        \item Проверяет отображение изображений на странице поста
    \end{itemize}
    
    \item \textbf{Редактирование поста:}
    \begin{itemize}
        \item Проверяет URL редактирования: \texttt{/posts/<post\_id>/edit/}
        \item Проверяет, что только автор может редактировать свой пост
        \item Проверяет, что другие пользователи не могут редактировать чужой пост
    \end{itemize}
    
    \item \textbf{Удаление поста:}
    \begin{itemize}
        \item Проверяет удаление поста автором
        \item Проверяет, что после удаления пост исчезает из БД
        \item Проверяет возврат 404 при обращении к удаленному посту
        \item Проверяет возврат 404 при обращении к несуществующему посту
    \end{itemize}
    
    \item \textbf{Права доступа:}
    \begin{itemize}
        \item Неопубликованные посты (\texttt{is\_published=False}) видны только автору
        \item Посты с неопубликованной категорией видны только автору
        \item Отложенные посты (с будущей датой) видны только автору
    \end{itemize}
\end{enumerate}

\textbf{Где реализовано:}
\begin{itemize}
    \item Создание: \texttt{blog/views.py::create\_post()}, строки 106-118
    \item Редактирование: \texttt{blog/views.py::edit\_post()}, строки 121-134
    \item Удаление: \texttt{blog/views.py::delete\_post()}, строки 137-147
    \item Детали: \texttt{blog/views.py::post\_detail()}, строки 30-51 (проверка прав доступа)
    \item Маршруты: \texttt{blog/urls.py}, строки 14-16
\end{itemize}

\subsection{Тесты модели Comment (test\_comment.py)}

\subsubsection{Тест 5-8: TestCommentModelAttrs (параметризованный класс)}
\textbf{Файл:} \texttt{tests/test\_comment.py}, строки 26-48

\textbf{Что проверяет:}
\begin{enumerate}
    \item \textbf{Поле \texttt{post}} --- ForeignKey к модели Post
    \item \textbf{Поле \texttt{author}} --- ForeignKey к модели User
    \item \textbf{Поле \texttt{text}} --- TextField для текста комментария
    \item \textbf{Поле \texttt{created\_at}} --- DateTimeField с \texttt{auto\_now\_add=True}
\end{enumerate}

\textbf{Где реализовано:}
\begin{itemize}
    \item Модель: \texttt{blog/models.py::Comment}, строки 118-142
    \item Поле \texttt{post}: строки 120-125
    \item Поле \texttt{author}: строки 126-130
    \item Поле \texttt{text}: строка 119
    \item Поле \texttt{created\_at}: строки 131-134
\end{itemize}

\subsubsection{Тест 9: test\_comment\_created\_at}
\textbf{Файл:} \texttt{tests/test\_comment.py}, строки 50-59

\textbf{Что проверяет:} Автоматическое заполнение \texttt{created\_at} при создании комментария.

\textbf{Где реализовано:}
\begin{itemize}
    \item Модель: \texttt{blog/models.py::Comment}, строка 131-134
    \item Параметр \texttt{auto\_now\_add=True}
\end{itemize}

\subsubsection{Тест 10: test\_comment (комплексный тест)}
\textbf{Файл:} \texttt{tests/test\_comment.py}, строки 101-248

\textbf{Что проверяет:} Полный цикл работы с комментариями.

\textbf{Как работает:}
\begin{enumerate}
    \item \textbf{Создание комментариев:}
    \begin{itemize}
        \item Проверяет, что неавторизованный пользователь не может создать комментарий
        \item Создает несколько комментариев
        \item Проверяет редирект на страницу поста после создания
        \item Проверяет отображение комментариев на странице поста
    \end{itemize}
    
    \item \textbf{Отображение количества комментариев:}
    \begin{itemize}
        \item Проверяет отображение количества в формате \texttt{(N)} на главной странице
        \item Проверяет отображение количества на странице профиля
    \end{itemize}
    
    \item \textbf{Редактирование комментария:}
    \begin{itemize}
        \item Проверяет URL: \texttt{/posts/<post\_id>/edit\_comment/<comment\_id>/}
        \item Проверяет, что только автор может редактировать свой комментарий
    \end{itemize}
    
    \item \textbf{Удаление комментария:}
    \begin{itemize}
        \item Проверяет удаление комментария автором
        \item Проверяет возврат 404 при обращении к удаленному комментарию
    \end{itemize}
\end{enumerate}

\textbf{Где реализовано:}
\begin{itemize}
    \item Создание: \texttt{blog/views.py::add\_comment()}, строки 150-159
    \item Редактирование: \texttt{blog/views.py::edit\_comment()}, строки 162-175
    \item Удаление: \texttt{blog/views.py::delete\_comment()}, строки 178-187
    \item Отображение количества: \texttt{blog/views.py::index()}, строка 22 (через \texttt{annotate(comment\_count=Count('comments'))})
    \item Маршруты: \texttt{blog/urls.py}, строки 17-22
\end{itemize}

\subsubsection{Тест 11: test\_404\_on\_comment\_deleted\_post}
\textbf{Файл:} \texttt{tests/test\_comment.py}, строки 250-301

\textbf{Что проверяет:} При попытке создать комментарий к несуществующему посту возвращается статус 404.

\textbf{Как работает:}
\begin{itemize}
    \item Удаляет пост
    \item Пытается создать комментарий к удаленному посту
    \item Проверяет, что возвращается статус 404
\end{itemize}

\textbf{Где реализовано:}
\begin{itemize}
    \item Представление: \texttt{blog/views.py::add\_comment()}, строка 152
    \item Использует \texttt{get\_object\_or\_404()} для получения поста
\end{itemize}

\subsection{Тесты пользователей (test\_users.py)}

\subsubsection{Тест 12: test\_custom\_err\_handlers}
\textbf{Файл:} \texttt{tests/test\_users.py}, строки 25-92

\textbf{Что проверяет:}
\begin{itemize}
    \item Подключение маршрутов из \texttt{django.contrib.auth.urls}
    \item Переопределение маршрута \texttt{auth/registration/}
    \item Наличие всех шаблонов авторизации в папке \texttt{templates/registration/}
    \item Корректность настройки \texttt{TEMPLATES\_DIR} в settings
\end{itemize}

\textbf{Где реализовано:}
\begin{itemize}
    \item Маршруты: \texttt{blogicum/urls.py}, строки 23, 25
    \item Регистрация: \texttt{blog/urls.py}, строка 23
    \item Представление: \texttt{blog/views.py::registration()}, строки 190-199
    \item Настройки: \texttt{blogicum/settings.py}, строка 57 (\texttt{TEMPLATES\_DIR})
    \item Шаблоны: \texttt{templates/registration/*.html}
\end{itemize}

\subsubsection{Тест 13: test\_profile}
\textbf{Файл:} \texttt{tests/test\_users.py}, строки 94-176

\textbf{Что проверяет:}
\begin{enumerate}
    \item \textbf{Страница профиля:}
    \begin{itemize}
        \item При обращении к несуществующему пользователю возвращается 404
        \item На странице отображается имя (\texttt{first\_name}) и фамилия (\texttt{last\_name})
        \item Страница доступна всем (авторизованным и неавторизованным)
    \end{itemize}
    
    \item \textbf{Редактирование профиля:}
    \begin{itemize}
        \item Ссылки ``Редактировать профиль'' и ``Изменить пароль'' видны только владельцу профиля
        \item Неавторизованные пользователи не видят ссылки редактирования
        \item Другие пользователи не видят ссылки редактирования
        \item Проверяет URL изменения пароля: \texttt{/auth/password\_change/}
    \end{itemize}
    
    \item \textbf{Редактирование данных:}
    \begin{itemize}
        \item Проверяет возможность редактирования профиля владельцем
        \item Проверяет, что неавторизованные пользователи не могут редактировать
    \end{itemize}
\end{enumerate}

\textbf{Где реализовано:}
\begin{itemize}
    \item Профиль: \texttt{blog/views.py::profile()}, строки 77-89
    \item Редактирование: \texttt{blog/views.py::edit\_profile()}, строки 92-103
    \item Шаблон: \texttt{templates/blog/profile.html}
    \item Форма: \texttt{blog/forms.py::UserForm}, строки 36-45
    \item Маршруты: \texttt{blog/urls.py}, строки 12-13
\end{itemize}

\subsection{Тесты статических страниц (test\_static\_pages.py)}

\subsubsection{Тест 14: test\_static\_pages\_as\_cbv}
\textbf{Файл:} \texttt{tests/test\_static\_pages.py}, строки 1-29

\textbf{Что проверяет:}
\begin{itemize}
    \item Статические страницы используют CBV (Class-Based Views)
    \item В \texttt{pages/urls.py} определен \texttt{app\_name}
    \item Маршруты подключены через \texttt{TemplateView.as\_view()}
\end{itemize}

\textbf{Где реализовано:}
\begin{itemize}
    \item Представления: \texttt{pages/views.py::AboutView}, \texttt{pages/views.py::RulesView}, строки 5-10
    \item Маршруты: \texttt{pages/urls.py}, строки 7-10
    \item Используется \texttt{TemplateView} из \texttt{django.views.generic}
\end{itemize}

\subsection{Тесты страниц ошибок (test\_err\_pages.py)}

\subsubsection{Тест 15: test\_csrf\_failure\_view}
\textbf{Файл:} \texttt{tests/test\_err\_pages.py}, строки 13-43

\textbf{Что проверяет:}
\begin{itemize}
    \item В \texttt{settings.py} задана настройка \texttt{CSRF\_FAILURE\_VIEW}
    \item View-функция существует и работает без ошибок
    \item Возвращает статус 403 при ошибке CSRF
\end{itemize}

\textbf{Где реализовано:}
\begin{itemize}
    \item Настройка: \texttt{blogicum/settings.py} (неявно через middleware)
    \item Обработчик: \texttt{blogicum/urls.py}, строка 33 (\texttt{handler403})
    \item Функция: \texttt{pages/views.py::csrf\_failure()}, строки 13-14
    \item Шаблон: \texttt{templates/pages/403csrf.html}
\end{itemize}

\subsubsection{Тест 16: test\_custom\_err\_handlers}
\textbf{Файл:} \texttt{tests/test\_err\_pages.py}, строки 46-125

\textbf{Что проверяет:}
\begin{enumerate}
    \item \textbf{Наличие шаблонов ошибок:}
    \begin{itemize}
        \item \texttt{404.html} для страницы не найдена
        \item \texttt{403csrf.html} для ошибки CSRF
        \item \texttt{500.html} для внутренней ошибки сервера
    \end{itemize}
    
    \item \textbf{Обработчики ошибок:}
    \begin{itemize}
        \item В \texttt{blogicum/urls.py} заданы \texttt{handler404}, \texttt{handler500}
        \item Обработчики указывают на существующие функции
        \item View-функции используют правильные шаблоны
    \end{itemize}
    
    \item \textbf{Работа 404:}
    \begin{itemize}
        \item При обращении к несуществующему URL используется шаблон \texttt{pages/404.html}
    \end{itemize}
\end{enumerate}

\textbf{Где реализовано:}
\begin{itemize}
    \item Обработчики: \texttt{blogicum/urls.py}, строки 33-35
    \item Функции: \texttt{pages/views.py}, строки 13-22
    \item Шаблоны: \texttt{templates/pages/404.html}, \texttt{templates/pages/403csrf.html}, \texttt{templates/pages/500.html}
\end{itemize}

\subsection{Тесты email (test\_emails.py)}

\subsubsection{Тест 17: test\_gitignore}
\textbf{Файл:} \texttt{tests/test\_emails.py}, строки 5-22

\textbf{Что проверяет:} Директория \texttt{sent\_emails/} указана в файле \texttt{.gitignore} в корне проекта.

\textbf{Где реализовано:}
\begin{itemize}
    \item Файл: \texttt{.gitignore} (в корне проекта)
    \item Должна быть строка: \texttt{sent\_emails/}
\end{itemize}

\subsubsection{Тест 18: test\_email\_backend\_settings}
\textbf{Файл:} \texttt{tests/test\_emails.py}, строки 25-37

\textbf{Что проверяет:}
\begin{itemize}
    \item В настройках задан \texttt{EMAIL\_BACKEND}
    \item Используется файловый бэкенд для отправки email
    \item В \texttt{EMAIL\_FILE\_PATH} указан путь \texttt{BASE\_DIR / 'sent\_emails'}
\end{itemize}

\textbf{Где реализовано:}
\begin{itemize}
    \item Настройки: \texttt{blogicum/settings.py}, строки 135-136
    \item \texttt{EMAIL\_BACKEND = 'django.core.mail.backends.filebased.EmailBackend'}
    \item \texttt{EMAIL\_FILE\_PATH = BASE\_DIR / 'sent\_emails'}
\end{itemize}

\subsection{Тесты контента (test\_content.py)}

\subsubsection{Тест 19: test\_unpublished}
\textbf{Файл:} \texttt{tests/test\_content.py}, строки 339-374

\textbf{Что проверяет:}
\begin{itemize}
    \item На странице профиля автор видит свои неопубликованные посты
    \item На главной странице не отображаются неопубликованные посты
    \item На странице категории не отображаются неопубликованные посты
\end{itemize}

\textbf{Где реализовано:}
\begin{itemize}
    \item Главная: \texttt{blog/views.py::index()}, строка 20 (\texttt{is\_published=True})
    \item Категория: \texttt{blog/views.py::category\_posts()}, строка 65 (\texttt{is\_published=True})
    \item Профиль: \texttt{blog/views.py::profile()}, строка 79 (нет фильтра по \texttt{is\_published}, показываются все посты автора)
\end{itemize}

\subsubsection{Тест 20: test\_only\_own\_pubs\_in\_category}
\textbf{Файл:} \texttt{tests/test\_content.py}, строки 376-390

\textbf{Что проверяет:} На странице категории отображаются только посты этой категории, посты других категорий не показываются.

\textbf{Где реализовано:}
\begin{itemize}
    \item Представление: \texttt{blog/views.py::category\_posts()}, строка 63
    \item Фильтр: \texttt{category=category} --- фильтрует посты по конкретной категории
\end{itemize}

\subsubsection{Тест 21: test\_only\_own\_pubs\_in\_profile}
\textbf{Файл:} \texttt{tests/test\_content.py}, строки 392-406

\textbf{Что проверяет:} На странице профиля пользователя отображаются только его посты, посты других авторов не показываются.

\textbf{Где реализовано:}
\begin{itemize}
    \item Представление: \texttt{blog/views.py::profile()}, строка 79-80
    \item Фильтр: \texttt{author=user} --- фильтрует посты по автору
\end{itemize}

\subsubsection{Тест 22: test\_unpublished\_category}
\textbf{Файл:} \texttt{tests/test\_content.py}, строки 408-438

\textbf{Что проверяет:}
\begin{itemize}
    \item На странице профиля автор видит свои посты с неопубликованной категорией
    \item На главной странице не отображаются посты с неопубликованной категорией
    \item Страница категории недоступна (404), если категория снята с публикации
\end{itemize}

\textbf{Где реализовано:}
\begin{itemize}
    \item Главная: \texttt{blog/views.py::index()}, строка 21 (\texttt{category\_\_is\_published=True})
    \item Категория: \texttt{blog/views.py::category\_posts()}, строка 58 (\texttt{is\_published=True} для категории)
    \item Профиль: \texttt{blog/views.py::profile()} --- нет фильтра по категории, показываются все посты автора
\end{itemize}

\subsubsection{Тест 23: test\_future\_posts}
\textbf{Файл:} \texttt{tests/test\_content.py}, строки 440-473

\textbf{Что проверяет:}
\begin{itemize}
    \item На странице профиля автор видит свои отложенные посты (с будущей датой)
    \item На главной странице не отображаются отложенные посты
    \item На странице категории не отображаются отложенные посты
\end{itemize}

\textbf{Где реализовано:}
\begin{itemize}
    \item Главная: \texttt{blog/views.py::index()}, строка 19 (\texttt{pub\_date\_\_lte=timezone.now()})
    \item Категория: \texttt{blog/views.py::category\_posts()}, строка 64 (\texttt{pub\_date\_\_lte=timezone.now()})
    \item Профиль: \texttt{blog/views.py::profile()} --- нет фильтра по дате, показываются все посты автора
    \item Детали поста: \texttt{blog/views.py::post\_detail()}, строки 40-43 (проверка доступа к отложенным постам)
\end{itemize}

\subsubsection{Тест 24: test\_pagination}
\textbf{Файл:} \texttt{tests/test\_content.py}, строки 475-545

\textbf{Что проверяет:}
\begin{enumerate}
    \item \textbf{Пагинация:}
    \begin{itemize}
        \item На странице профиля работает пагинация (10 постов на страницу)
        \item На главной странице работает пагинация
        \item На странице категории работает пагинация
    \end{itemize}
    
    \item \textbf{Сортировка:}
    \begin{itemize}
        \item Посты отсортированы по дате публикации ``от новых к старым'' (\texttt{-pub\_date})
        \item Сортировка работает на всех страницах
    \end{itemize}
\end{enumerate}

\textbf{Где реализовано:}
\begin{itemize}
    \item Главная: \texttt{blog/views.py::index()}, строки 22-25
    \begin{itemize}
        \item Сортировка: \texttt{order\_by('-pub\_date')}, строка 22
        \item Пагинация: \texttt{Paginator(post\_list, 10)}, строка 23
    \end{itemize}
    \item Категория: \texttt{blog/views.py::category\_posts()}, строки 66-69
    \begin{itemize}
        \item Сортировка: \texttt{order\_by('-pub\_date')}, строка 66
        \item Пагинация: \texttt{Paginator(post\_list, 10)}, строка 67
    \end{itemize}
    \item Профиль: \texttt{blog/views.py::profile()}, строки 81-84
    \begin{itemize}
        \item Сортировка: \texttt{order\_by('-pub\_date')}, строка 81
        \item Пагинация: \texttt{Paginator(post\_list, 10)}, строка 82
    \end{itemize}
    \item Шаблон пагинации: \texttt{templates/includes/paginator.html}
\end{itemize}

\subsubsection{Тест 25: test\_image\_visible}
\textbf{Файл:} \texttt{tests/test\_content.py}, строки 547-579

\textbf{Что проверяет:}
\begin{itemize}
    \item На главной странице отображаются изображения постов
    \item На странице профиля отображаются изображения постов
    \item На странице категории отображаются изображения постов
    \item Если у поста нет изображения, оно не отображается
\end{itemize}

\textbf{Как работает:}
\begin{itemize}
    \item Создает пост с изображением
    \item Подсчитывает количество тегов \texttt{<img>} на каждой странице
    \item Удаляет изображение у поста
    \item Проверяет, что количество изображений уменьшилось на 1
\end{itemize}

\textbf{Где реализовано:}
\begin{itemize}
    \item Модель: \texttt{blog/models.py::Post}, поле \texttt{image}, строки 103-107
    \item Шаблоны:
    \begin{itemize}
        \item \texttt{templates/includes/post\_card.html} --- карточка поста с изображением
        \item \texttt{templates/blog/index.html} --- главная страница
        \item \texttt{templates/blog/profile.html} --- страница профиля
        \item \texttt{templates/blog/category.html} --- страница категории
    \end{itemize}
    \item Отображение: через шаблон \texttt{post\_card.html}, который включается на всех страницах
\end{itemize}

\subsection{Сводная таблица всех тестов}

\begin{table}[H]
\centering
\small
\begin{tabular}{|p{1.5cm}|p{2cm}|p{4cm}|p{5cm}|}
\hline
\textbf{№} & \textbf{Файл} & \textbf{Название теста} & \textbf{Что проверяет} \\
\hline
1-2 & test\_post.py & TestPostModelAttrs & Поля модели Post (image, pub\_date) \\
\hline
3 & test\_post.py & test\_post\_created\_at & Автозаполнение created\_at \\
\hline
4 & test\_post.py & test\_post & Полный цикл работы с постами \\
\hline
5-8 & test\_comment.py & TestCommentModelAttrs & Поля модели Comment \\
\hline
9 & test\_comment.py & test\_comment\_created\_at & Автозаполнение created\_at \\
\hline
10 & test\_comment.py & test\_comment & Полный цикл работы с комментариями \\
\hline
11 & test\_comment.py & test\_404\_on\_comment\_deleted\_post & 404 при комментарии к удаленному посту \\
\hline
12 & test\_users.py & test\_custom\_err\_handlers & Маршрутизация и шаблоны авторизации \\
\hline
13 & test\_users.py & test\_profile & Профиль пользователя и редактирование \\
\hline
14 & test\_static\_pages.py & test\_static\_pages\_as\_cbv & Использование CBV для статических страниц \\
\hline
15 & test\_err\_pages.py & test\_csrf\_failure\_view & Обработчик ошибки CSRF \\
\hline
16 & test\_err\_pages.py & test\_custom\_err\_handlers & Обработчики ошибок 404, 500, 403 \\
\hline
17 & test\_emails.py & test\_gitignore & Настройка .gitignore \\
\hline
18 & test\_emails.py & test\_email\_backend\_settings & Настройки email backend \\
\hline
19 & test\_content.py & test\_unpublished & Отображение неопубликованных постов \\
\hline
20 & test\_content.py & test\_only\_own\_pubs\_in\_category & Фильтрация по категории \\
\hline
21 & test\_content.py & test\_only\_own\_pubs\_in\_profile & Фильтрация по автору \\
\hline
22 & test\_content.py & test\_unpublished\_category & Посты с неопубликованной категорией \\
\hline
23 & test\_content.py & test\_future\_posts & Отложенные посты \\
\hline
24 & test\_content.py & test\_pagination & Пагинация и сортировка \\
\hline
25 & test\_content.py & test\_image\_visible & Отображение изображений \\
\hline
\end{tabular}
\caption{Сводная таблица всех 25 тестов проекта}
\end{table}

\section{Заключение}

Проект \textbf{Blogicum} представляет собой полнофункциональное веб-приложение блога с следующими возможностями:

\begin{itemize}
    \item Публикация постов с изображениями
    \item Категоризация и привязка к местоположениям
    \item Система комментариев
    \item Регистрация и авторизация пользователей
    \item Редактирование профиля
    \item Административная панель для управления контентом
    \item Пагинация для больших списков
    \item Обработка ошибок (404, 500, CSRF)
    \item Адаптивный дизайн с Bootstrap
\end{itemize}

Проект следует лучшим практикам Django и имеет четкую структуру, что облегчает его поддержку и развитие.

\end{document}

